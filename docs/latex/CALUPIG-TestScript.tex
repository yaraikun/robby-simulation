\documentclass[10pt, letterpaper]{article}
\usepackage{geometry}
\usepackage{array}
\usepackage{fontspec}
\usepackage{enumitem}
\usepackage{xcolor}
\usepackage{tabularx}

\geometry{margin=0.5in}
\setmainfont{Aptos}
\pagenumbering{gobble}

\begin{document}

\begin{flushleft}
    **Test Script should be in a table format, with header as shown below.
    There should be \textbf{at least 3 distinct test classes} (as indicated in
    the description) \textbf{per function}.
\end{flushleft}

\begin{flushleft}
    Test descriptions are supposed to be unique and should indicate classes/
    groups of test cases on what is being tested.  For example, given the 
    function getAreaTri() which computes the area of a triangle given the base
    and height as parameters, the following are 3 distinct classes of tests:
\end{flushleft}

\begin{itemize}[itemsep=-0.2em]
    \item testing with base and height values smaller than 1
    \item testing with whole number values for base and height
    \item testing with floating-point number values for base and height, larger
          than 1
\end{itemize}

\begin{flushleft}
    The following test descriptions are incorrectly formed:\\
          Too specific: testing with base containing 0.25 and height containing
                        0.75\\
          Too general: testing if function can generate correct area of
                       triangle\\
          Not necessary: since already defined in pre-condition: testing with
                         base or height containing negative values
\end{flushleft}

\begin{flushleft}
    \textcolor{red} {
        FILL-UP THE FOLLOWING TABLE (the 1st three rows are just examples; delete
        it in your own document).  Add new rows as you deem necessary…
    }
\end{flushleft}

\begin{table}[h]
    \centering
    \renewcommand{\arraystretch}{1.5}
    \begin{tabularx}{\textwidth}{
            |>{\raggedright\arraybackslash}p{3cm}
            |>{\raggedright\arraybackslash}p{0.20cm}
            |>{\raggedright\arraybackslash}p{2.5cm}
            |>{\raggedright\arraybackslash}X
            |>{\raggedright\arraybackslash}p{3cm}
            |>{\raggedright\arraybackslash}p{3cm}
            |>{\raggedright\arraybackslash}p{1cm}|
        }
        \hline

        \textbf{Function Name} &
        \textbf{\#} &
        \textbf{Test Description} &
        \textbf{Sample Input} &
        \textbf{Expected Result} &
        \textbf{Actual Result} &
        \textbf{Pass or Fail?} \\ 
        \hline

        InitializeReset &
        1 &
        pfRobotX, pfRobotY, and pdRobotAngle are positive numbers &
        pfRobotX = 100, pfRobotY = 100, pdRobotAngle = 100 &
        pfRObotX = 0, pfRobotY = 0, pdRobotAngle = 0 &
        pfRObotX = 0, pfRobotY = 0, pdRobotAngle = 0 &
        Pass \\
        \hline

        InitializeReset &
        2 &
        pfRobotX, pfRobotY, and pdRobotAngle are negative numbers &
        pfRobotX = -100, pfRobotY = -100, pdRobotAngle = -100 &
        pfRObotX = 0, pfRobotY = 0, pdRobotAngle = 0 &
        pfRObotX = 0, pfRobotY = 0, pdRobotAngle = 0 &
        Pass \\
        \hline

        InitializeReset &
        3 &
        pfRobotX, pfRobotY, and pdRobotAngle are zero &
        pfRobotX = 0, pfRobotY = 0, pdRobotAngle = 0 &
        pfRObotX = 0, pfRobotY = 0, pdRobotAngle = 0 &
        pfRObotX = 0, pfRobotY = 0, pdRobotAngle = 0 &
        Pass \\
        \hline

        TranslateForward &
        1 &
        fDistance and dRobotAngle are positive &
        fDistance = 100, pfRobotX = 0, pfRobotY = 0, pdRobotAngle = 45 &
        pfRobotX = 70.7107..., pfRobotY = 70.7107... &
        pfRobotX = 70.7107..., pfRobotY = 70.7107... &
        Pass \\
        \hline

        TranslateForward &
        2 &
        fDistance is zero and dRobotAngle is positive &
        fDistance = 0, pfRobotX = 0, pfRobotY = 0, pdRobotAngle = 45 &
        pfRobotX = 0.0..., pfRobotY = 0.0... &
        pfRobotX = 0.0..., pfRobotY = 0.0... &
        Pass \\
        \hline

        TranslateForward &
        3 &
        fDistance is negative and dRobotAngle is positive &
        fDistance = -100, pfRobotX = 0, pfRobotY = 0, pdRobotAngle = 45 &
        pfRobotX = -70.7107..., pfRobotY = -70.7107... &
        pfRobotX = -70.7107..., pfRobotY = -70.7107... &
        Pass \\
        \hline


    \end{tabularx}
\end{table}
\newpage

\begin{table}[h]
    \centering
    \renewcommand{\arraystretch}{1.5}
    \begin{tabularx}{\textwidth}{
            |>{\raggedright\arraybackslash}p{3cm}
            |>{\raggedright\arraybackslash}p{0.20cm}
            |>{\raggedright\arraybackslash}p{2.5cm}
            |>{\raggedright\arraybackslash}X
            |>{\raggedright\arraybackslash}p{3cm}
            |>{\raggedright\arraybackslash}p{3cm}
            |>{\raggedright\arraybackslash}p{1cm}|
        }
        \hline

        \textbf{Function Name} &
        \textbf{\#} &
        \textbf{Test Description} &
        \textbf{Sample Input} &
        \textbf{Expected Result} &
        \textbf{Actual Result} &
        \textbf{Pass or Fail?} \\ 
        \hline


        TranslateForward &
        4 &
        fDistance is positive and dRobotAngle is negative. &
        fDistance = 100, pfRobotX = 0, pfRobotY = 0, pdRobotAngle = -45 &
        pfRobotX = 70.7107..., pfRobotY = -70.7107... &
        pfRobotX = 70.7107..., pfRobotY = -70.7107... &
        Pass \\
        \hline

        TranslateForward &
        5 &
        fDistance is zero and dRobotAngle is positive. &
        fDistance = 0, pfRobotX = 0, pfRobotY = 0, pdRobotAngle = -45 &
        pfRobotX = 0.0... pfRobotY = 0.0... &
        pfRobotX = 0.0... pfRobotY = 0.0... &
        Pass \\
        \hline

        TranslateForward &
        6 &
        fDistance and dRobotAngle are negative. &
        fDistance = -100, pfRobotX = 0, pfRobotY = 0, pdRobotAngle = -45 &
        pfRobotX = -70.7107..., pfRobotY = 70.7107... &
        pfRobotX = -70.7107..., pfRobotY = 70.7107... &
        Pass \\
        \hline

        TranslateForward &
        7 &
        fDistance is positive and dRobotAngle is zero. &
        fDistance = 100, pfRobotX = 0, pfRobotY = 0, pdRobotAngle = 0 &
        pfRobotX = 100.0..., pfRobotY = 0.0... &
        pfRobotX = 100.0..., pfRobotY = 0.0... &
        Pass \\
        \hline

        TranslateForward &
        8 &
        fDistance and dRobotAngle are zero. &
        fDistance = 0, pfRobotX = 0, pfRobotY = 0, pdRobotAngle = 0 &
        pfRobotX = 0.0..., pfRobotY = 0.0... &
        pfRobotX = 0.0..., pfRobotY = 0.0... &
        Pass \\
        \hline

        TranslateForward &
        9 &
        fDistance is negative and dRobotAngle is zero. &
        fDistance = -100, pfRobotX = 0, pfRobotY = 0, pdRobotAngle = 0 &
        pfRobotX=-100.0..., pfRobotY = 0.0... &
        pfRobotX=-100.0..., pfRobotY = 0.0... &
        Pass \\
        \hline

        TranslateBackward &
        1 &
        fDistance and dRobotAngle is positive &
        fDistance = 100, pfRobotX = 0, pfRobotY = 0, pdRobotAngle = 45 &
        pfRobotX = -70.7107..., pfRobotY = -70.7107... &
        pfRobotX = -70.7107..., pfRobotY = -70.7107... &
        Pass \\
        \hline

        TranslateBackward &
        2 &
        fDistance is zero and dRobotAngle is positive &
        fDistance = 0, pfRobotX = 0, pfRobotY = 0, pdRobotAngle = 45 &
        pfRobotX = 0.0..., pfRobotY = 0.0... &
        pfRobotX = 0.0... pfRobotY = 0.0... &
        Pass \\
        \hline

        TranslateBackward &
        3 &
        fDistance is negative and dRobotAngle is positive &
        fDistance = -100, pfRobotX = 0, pfRobotY = 0, pdRobotAngle = 45 &
        pfRobotX = 70.7107..., pfRobotY = 70.7107... &
        pfRobotX = 70.7107..., pfRobotY = 70.7107... &
        Pass \\
        \hline

    \end{tabularx}
\end{table}
\newpage

\begin{table}[h]
    \centering
    \renewcommand{\arraystretch}{1.5}
    \begin{tabularx}{\textwidth}{
            |>{\raggedright\arraybackslash}p{3cm}
            |>{\raggedright\arraybackslash}p{0.20cm}
            |>{\raggedright\arraybackslash}p{2.5cm}
            |>{\raggedright\arraybackslash}X
            |>{\raggedright\arraybackslash}p{3cm}
            |>{\raggedright\arraybackslash}p{3cm}
            |>{\raggedright\arraybackslash}p{1cm}|
        }
        \hline

        \textbf{Function Name} &
        \textbf{\#} &
        \textbf{Test Description} &
        \textbf{Sample Input} &
        \textbf{Expected Result} &
        \textbf{Actual Result} &
        \textbf{Pass or Fail?} \\ 
        \hline

        TranslateBackward &
        4 &
        fDistance is positive and dRobotAngle is negative. &
        fDistance = 100, pfRobotX = 0, pfRobotY = 0, pdRobotAngle = -45 &
        pfRobotX = -70.7107..., pfRobotY = 70.7107... &
        pfRobotX = -70.7107..., pfRobotY = 70.7107... &
        Pass \\
        \hline

        TranslateBackward &
        5 &
        fDistance is zero and dRobotAngle is negative. &
        fDistance = 0, pfRobotX = 0, pfRobotY = 0, pdRobotAngle = -45 &
        pfRobotX = 0.0..., pfRobotY = 0.0... &
        pfRobotX = 0.0... pfRobotY = 0.0... &
        Pass \\
        \hline

        TranslateBackward &
        6 &
        fDistance is negative and dRobotAngle is negative. &
        fDistance = 0, pfRobotX = 0, pfRobotY = 0, pdRobotAngle = -45 &
        pfRobotX = 70.7107..., pfRobotY = -70.7107... &
        pfRobotX = 70.7107..., pfRobotY = -70.7107... &
        Pass \\
        \hline

        TranslateBackward &
        7 &
        fDistance is positive and dRobotAngle is zero. &
        fDistance = 100, pfRobotX = 0, pfRobotY = 0, pdRobotAngle = 0 &
        pfRobotX=-100.0..., pfRobotY = 0.0... &
        pfRobotX=-100.0..., pfRobotY = 0.0... &
        Pass \\
        \hline

        TranslateBackward &
        8 &
        fDistance and dRobotAngle are zero. &
        fDistance = 0, pfRobotX = 0, pfRobotY = 0, pdRobotAngle = 0 &
        pfRobotX = 0.0..., pfRobotY = 0.0
        pfRobotX = 0.0..., pfRobotY = 0.0... &
        pfRobotX = 0.0, pfRobotY = 0.0 &
        Pass \\
        \hline

        TranslateBackward &
        9 &
        fDistance is negative and dRobotAngle is zero. &
        fDistance = 100, pfRobotX = 0, pfRobotY = 0, pdRobotAngle = 0 &
        pfRobotX = 100.0..., pfRobotY = 0.0... &
        pfRobotX = 100.0..., pfRobotY = 0.0... &
        Pass \\
        \hline

        RotateCounter- Clockwise &
        1 &
        dTheta is positive. &
        dTheta = 10, pdRobotAngle = 80 &
        pdRobotAngle = 90.0... &
        pdRobotAngle = 90.0... &
        Pass \\
        \hline

        RotateCounter- Clockwise &
        2 &
        dTheta is zero. &
        dTheta = 0, pdRobotAngle = 80 &
        pdRobotAngle = 80.0... &
        pdRobotAngle = 80.0... &
        Pass \\
        \hline

        RotateCounter- Clockwise &
        3 &
        dTheta is negative. &
        dTheta = -10, pdRobotAngle = 80 &
        pdRobotAngle = 70.0... &
        pdRobotAngle = 70.0... &
        Pass \\
        \hline

        RotateClockwise &
        1 &
        dTheta is positive. &
        dTheta = 10, pdRobotAngle = 80 &
        pdRobotAngle = 70.0... &
        pdRobotAngle = 70.0... &
        Pass \\
        \hline

        RotateClockwise &
        2 &
        dTheta is zero. &
        dTheta = 0, pdRobotAngle = 80 &
        pdRobotAngle = 80.0... &
        pdRobotAngle = 80.0... &
        Pass \\
        \hline

        RotateClockwise &
        3 &
        dTheta is negative. &
        dTheta = -10, pdRobotAngle = 80 &
        pdRobotAngle = 90.0... &
        pdRobotAngle = 90.0... &
        Pass \\
        \hline

    \end{tabularx}
\end{table}

\end{document}
